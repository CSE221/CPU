\begin{abstract}
In building an operating system, it is important to be able to determine the performance characteristics of underlying hardware components (CPU, RAM, disk, network, etc.), and to understand how their performance influences or constrains operating system services. Likewise, in building an application, one should understand the performance of the underlying hardware and operating system, and how they relate to the user's subjective sense of that application's ``responsiveness''. While some of the relevant quantities can be found in specs and documentation, many must be determined experimentally. While some values may be used to predict others, the relations between lower-level and higher-level performance are often subtle and non-obvious.

In this project, we will create, justify, and apply a set of experiments to a system to characterize and understand its performance. In addition, we also explore the relations between some of these quantities. One of the goal of this project it that we will study how to use benchmarks to usefully characterize a complex system, and also we will also gain an intuitive feel for the relative speeds of different basic operations, which is invaluable in identifying performance bottlenecks.
%% \vspace{+0.1in}
\end{abstract}